\documentclass[11pt]{article}
\usepackage[utf8]{inputenc}
\usepackage{geometry}
\geometry{a4paper}


\title{SOAS Exchange App: Week 5 Update}
\author{Louis Fillo, John Hill & Gal Moore}

\begin{document}
\maketitle

\section{Progress overview}
Since the last report, we have met  and agreed upon many design decisions. Our wireframe was a good starting point which we agreed will be built upon. Out client was happy with the design. 

\section{Client Requirements}
\begin{itemize}
   \item Based on Tinder.
   \item Users create a profile with 'What languages I would like to learn' and 'What langauges I have to offer'.
   \item Both 'What I would like' and 'What I have to offer' have the same categories of information.
   \item Standardised list of qualities to match on.
   \item Option to 'star' particular qualities in the profile, which increases the chances that matches with a user that possesses this quality.
   \item Each user matches with another user who possess the qualities that they would like to develop.
   \item This 'wanting' user is the able to message this 'possessing' user to talk about how they could help them get this skill.
\end{itemize}

\subsection{Applets}
\subsubsection{Language Exchange}
Facilitates the matching of users (predominantly SOAS students) for the purpose of language exchange.  Language exchange involves two participants meeting up to speak in languages that one side is a fluent/native speaker of, and the other user is aiming to learn.  The exchange is reciprocal: exchangers swap roles.  The SOAS community covers a wide range of languages, so the app would need features to match users who speak more obscure languages, as well as the major languages.

\subsubsection{Almuni}
Facilitates matching of current students with opportunities offered by alumni.  For example, an almunus may require an individual fluent in Japanese and English for a short placement in Tokyo.

\subsubsection{Impact}
Facilitates the matching of academics with other academics who have experience of cultures/institutions/situations that less experienced colleagues may be about to encounter for the first time.  For example, a junior academic may have been requested to speak in front of a select committee for the first time.  They may want to find a more experienced colleague to discuss the experience with.

\subsection{Summary of meeting (17/11)
We had very productive meeting with our client.  ith the direction of the project and the wirefarmes presented.  A number of small changes were presented.  These included:


\section{Individual actions/Completed Tasks}

\subsection{Gal Moore}
Client Liaison \& UX design
\begin{itemize}
   \item Liaised with client.  
   \item Developed preliminary storyboards (User stories)
\end{itemize}

\subsection{Louis Fillo}
Programmer \& Tester
\begin{itemize}
   \item Set up SOAS meeting
   \item Worked on wireframe
\end{itemize}

\subsection{John Hill}
Researcher \& Editor
\begin{itemize}
   \item Written questions for User questionnaire.  
   \item Initiated competitor analysis -- researching language exchange apps (e.g. HelloTalk, LextTalk, Tandem)
\end{itemize}

\section{Issues Encountered}
\begin{itemize}
   \item User interview questions need further refinement: current interview structure has not gleaned enough information to inform design choices.
\end{itemize}

\section{Next Steps}
\begin{itemize}
   \item Start developing final product
   \item Work out best software to develop the app
   \item Designate workload between team members
   \item Take on SOAS's comments and implement them into our design
\end{itemize}

\end{document}
\medskip




